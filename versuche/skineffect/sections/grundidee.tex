Hochfrequente  Wechselstr\"ome  haben  die   Eigenschaft,  dass  sie  v.a.  an
der  Oberfl\"ache  eines  Leiters  fliessen  und  nicht  tief  in  den  Leiter
eindringen. Dieses als  \emph{Skineffekt} bekannte  Ph\"anomen soll  in diesem
Versuch experimentell nachgewiesen werden.

Wird  ein  Leiter  in  ein  wechselndes  Magnetfeld  eingef\"uhrt,  werden  in
ihm  Wirbelstr\"ome  induziert. Ist  die  Frequenz  des  externen  Magnetfelds
niedrig,  verteilen  sich  diese  Wirbelstr\"ome  (ungleichm\"assig)  auf  den
gesamten  Querschnitt.  Bei  h\"oheren  Frequenzen  des externen  Magnetfeldes
verlagern   sich   die   Wirbelstr\"ome  in   den   Oberfl\"achenbereich   des
Leiters. Da sie  der \"Anderung  des externen  Feldes gem\"ass  der Lenz'schen
Regel~\cite{ref:wikipedia:lenzscheRegel}  entgegenwirken,  schw\"achen sie  im
Innern  des  Leiters  das  externe  Feld  ab. Ebenfalls  werden  der  Ohm'sche
Widerstand und der Selbstinduktionskoeffizient der Konfiguration von Spule und
Zylinder ge\"andert, sowie der magnetische Fluss im Innern des Zylinders.

Als Versuchsobjekte dienen die  F\"alle eines eingef\"uhrten Hohlzylinders und
eines  eingef\"uhrten Vollzylinders. Es werden  sowohl g\"angige  N\"aherungen
wie  auch die  exakten  L\"osungen  aus der  Theorie  mit den  Messergebnissen
verglichen.

Neben den Plots  und zuge\"origen Beobachtungen \"uber den  Skineffekt und die
Abschirmung des B-Feldes im Innern  des eingef\"uhrten Leiters kann aus diesem
Versuch noch die  Leitf\"ahgigkeit des Materials des  Leiters bestimmt werden,
da sie in den Funktionen zur  Beschreibung des B-Feldes vorkommt und somit ein
Parameter des Fit-Vorgangs ist. Wie man  sehen wird, weichen die experimentell
ermittelten Werte teilweise bedeutend von den Literaturwerten ab.
