Da der Punkt dieses Versuches die Fehlerrechnung selbst ist, beinhaltet dieses
Kapitel  ausnahmsweise  auch  die  Fehlerrechnung. \"Ublicherweise  ist  diese
jedoch in einem separaten Kapitel zu finden.

% **************************************************************************** %
\subsection{Aufgabe 1: Schallgeschwindigkeit}
% **************************************************************************** %

\subsubsection{Daten}

\begin{itemize}
    \item
        \emph{L\"ange der Messtrecke}: $\SI[separate-uncertainty = true]{2.561 \pm 0.003 }{\meter}$
    \item
        \emph{Raumtemperatur}: $\vartheta = \SI{23}{\celsius}$
\end{itemize}

Messprotokoll:

\begin{center}
\begin{tabular}{rr|rr}
    \toprule
    Messung & Laufzeit $t_i$ (ms) & Messung & Laufzeit $t_i$ (ms) \\
    \midrule
     1       & 6.83  & 11       & 7.36 \\
     2       & 7.41  & 12       & 7.31 \\
     3       & 7.32  & 13       & 7.56 \\
     4       & 7.31  & 14       & 7.14 \\
     5       & 7.23  & 15       & 6.94 \\
     6       & 7.68  & 16       & 7.32 \\
     7       & 7.33  & 17       & 7.34 \\
     8       & 7.7   & 18       & 7.28 \\
     9       & 7.93  & 19       & 7.01 \\
    10       & 7.54  & 20       & 7.76 \\
    \bottomrule
\end{tabular}
\end{center}


\subsubsection{Mittlere Laufzeit und ihre Unsicherheit}

Mittlere Laufzeit:
\begin{equation}
    \overline{t} = \frac{1}{20} \sum_{i=1}^{20} t_i = \SI{7.37}{\milli\second}
\end{equation}
Fehler des Mittelwertes:
\begin{equation}
    %s_{\overline{t}} = \frac{1}{20} \sum_{i=1}^{20} t_i = \SI{0.01}{\milli\second}
    s_{\overline{t}} = \sqrt{ \frac{\sum_{1}^{20}{(t_i-\overline{t})^2}}{20 \cdot 19}} = \SI{0.062}{\milli\second} \\
\end{equation}

\subsubsection{Wert und Unsicherheit der Schallgeschwindigkeit}

Formel f\"ur Schallgeschwindigkeit in trockener Luft um $\SI{0}{\celsius}$:
%(Quelle: https://de.wikipedia.org/wiki/Schallgeschwindigkeit, https://en.wikipedia.org/wiki/Speed_of_sound)

\begin{equation}
    c_{luft} = (331.3 + 0.606 \cdot \vartheta) \si{\meter\per\second} = (331.3 + 0.606 \cdot 23) \si{\meter\per\second} = \SI{345.24}{\meter\per\second}
\end{equation}

Berechnung der mittleren Geschwindigkeit:

\begin{gather}
    c = \frac{s}{t} \\
    \overline{c} = \frac{s}{\overline{t}} = \frac{\SI{2.561}{\meter}}{\SI{7.37}{\milli\second}} = \SI{349.74}{\meter\per\second} \\
\end{gather}

Gauss'sches Fehlerfortpflanzungsgesetz:
\begin{equation}
    s_{\overline{R}} = \sqrt{ \left( \frac{\partial R}{\partial x} \biggr\rvert_{\overline{R}} \cdot s_{\overline{x}}\right)^2
                            + \left( \frac{\partial R}{\partial y} \biggr\rvert_{\overline{R}} \cdot s_{\overline{y}}\right)^2
                            + \left( \frac{\partial R}{\partial z} \biggr\rvert_{\overline{R}} \cdot s_{\overline{z}}\right)^2
                            + ... }
\end{equation}

In diesem Fall ist $R(x,y,z,...) := c(s,t) = \frac{s}{t}$. Es ergibt sich die Formel:
\begin{gather*}
    s_{\overline{c(s,t)}} = \sqrt{ \left( \frac{\partial c}{\partial s} \biggr\rvert_{\overline{c}} \cdot s_{\overline{s}}\right)^2
                            + \left( \frac{\partial c}{\partial t} \biggr\rvert_{\overline{c}} \cdot s_{\overline{t}}\right)^2 } \\
                            = \sqrt{ \left( \frac{\partial}{\partial s}\frac{s}{t} \biggr\rvert_{\overline{c}} \cdot s_{\overline{s}}\right)^2
                            + \left( \frac{\partial}{\partial t}\frac{s}{t} \biggr\rvert_{\overline{c}} \cdot s_{\overline{t}}\right)^2 } \\
                            = \sqrt{ \left( \frac{1}{t} \biggr\rvert_{\overline{c}} \cdot s_{\overline{s}}\right)^2
                            + \left( - \frac{s}{t^2} \biggr\rvert_{\overline{c}} \cdot s_{\overline{t}}\right)^2 } \\
                            = \sqrt{ \left( \frac{1}{\overline{t}} \cdot s_{\overline{s}}\right)^2
                            + \left( - \frac{\overline{s}}{\overline{t}^2} \cdot s_{\overline{t}}\right)^2 } \\
                            = \sqrt{ \left( \frac{1}{\SI{7.37}{\milli\second}} \cdot \SI{3}{\milli\meter} \right)^2
                            + \left( - \frac{\SI{2.561}{\meter}}{(\SI{7.37}{\milli\second})^2} \cdot \SI{0.062}{\milli\second}\right)^2 } \\
                            = \sqrt{ \left( \frac{1}{\SI{0.00737}{\second}} \cdot \SI{0.003}{\meter} \right)^2
                            + \left( - \frac{\SI{2.561}{\meter}}{(\SI{0.00737}{\second})^2} \cdot \SI{0.000062}{\second}\right)^2 } \\
                            = \SI{2.93}{\meter\per\second} \text{ (Resultat von Tabellenkalkulationsprogramm)} \\
                            = \SI{2.95}{\meter\per\second} \text{ (Resultat mittels Eintippen der obigen Zahlen in Taschenrechner)} \\
\end{gather*}

Folglich:
\begin{equation}
    c_{luft} = \overline{c_{luft}} \pm s_{\overline{c_{luft}}} = \SI[separate-uncertainty = true]{350 \pm 3}{\meter\per\second}
\end{equation}

\clearpage
% **************************************************************************** %
\subsection{Aufgabe 2: Eisengehalt}
% **************************************************************************** %

\subsubsection{Daten}
\begin{center}
\begin{tabular}{rrr}
    \toprule
    Messung & Eisengehalt (\%) & absoluter Fehler (\%) \\
    \midrule
    1 & 20.3 & 1.2 \\
    2 & 21.9 & 1.3 \\
    3 & 21.1 & 1.1 \\
    4 & 19.6 & 0.8 \\
    5 & 19.9 & 1.3 \\
    6 & 18.0 & 1.3 \\
    7 & 19.4 & 1.0 \\
    8 & 22.2 & 2.0 \\
    9 & 21.6 & 0.8 \\
    \bottomrule
\end{tabular}
\end{center}

\subsubsection{Einfacher Mittelwert}

Der einfache Mittelwert ergibt sich als:
\begin{equation}
    \overline{x} = \frac{1}{9}\sum_{i=1}^9 x_i = \SI{20.44}{\percent}
\end{equation}

Mit dem zugeh\"origen Fehler:
\begin{equation}
    s_{\overline{x}} = \sqrt{\frac{\sum_1^9(x_i-\overline{x})^2}{9 \cdot 8}} = \SI{0.46}{\percent}
\end{equation}


\subsubsection{Gewichteter Mittelwert}

Der gewichtete Mittelwert errechnet sich gem\"ass:
\begin{equation}
    \overline{x} = \frac{\sum_1^9 g_{\overline{x_i}} \cdot x_i}{\sum_1^9 g_{\overline{x_i}}} = \frac{156.24}{7.67} \si{\percent} = \SI{20.37}{\percent}
\end{equation}

Der zugeh\"orige Fehler Betr\"agt:
\begin{equation}
    s_{\overline{x}} = \frac{1}{\sum_1^9 g_{\overline{x_i}}} = \SI{0.36}{\percent}
\end{equation}


\clearpage
% **************************************************************************** %
\subsection{Aufgabe 3: Federkonstante}
% **************************************************************************** %

\subsubsection{Daten}

\begin{center}
\begin{tabular}{rr}
    \toprule
    F (N) & z (m) \\
    \midrule
    3.83  & 0.20 \\
    7.79  & 0.35 \\
    8.08  & 0.42 \\
    9.7   & 0.46 \\
    10.58 & 0.51 \\
    12.33 & 0.54 \\
    12.23 & 0.59 \\
    14.43 & 0.67 \\
    15.51 & 0.71 \\
    17.09 & 0.80 \\
    \bottomrule
\end{tabular}
\end{center}

\subsubsection{Rechnung mittels Tabellenkalkulation}

\begin{tabular}{rrrrrrrr}
    \toprule
                F (N)&z (m)&&&&&& $\hat{F}(N)$ \\
    \midrule
             $y_i$      & $x_i$      &  $y_i-\overline{y}$ & $x_i-\overline{x} $ & $(y_i-\overline{y})(x_i-\overline{x})$ & $(y_i-\overline{y})^2$ & $(x_i-\overline{x})^2$ & $ \hat{y}  $ \\
    \midrule
               3.83&        0.20&              -7.33  &              -0.32  &                 2.38                   &        53.68           &         0.11           &  3.82 \\
               7.79&        0.35&              -3.37  &              -0.17  &                 0.59                   &        11.34           &         0.03           &  7.21 \\
               8.08&        0.42&              -3.08  &              -0.10  &                 0.32                   &         9.47           &         0.01           &  8.79 \\
               9.70&        0.46&              -1.46  &              -0.06  &                 0.09                   &         2.12           &         0.00           &  9.69 \\
              10.58&        0.51&              -0.58  &              -0.01  &                 0.01                   &         0.33           &         0.00           &  10.82 \\
              12.33&        0.54&               1.17  &               0.02  &                 0.02                   &         1.38           &         0.00           &  11.50 \\
              12.23&        0.59&               1.07  &               0.07  &                 0.07                   &         1.15           &         0.00           &  12.62 \\
              14.43&        0.67&               3.27  &               0.15  &                 0.47                   &        10.71           &         0.02           &  14.43 \\
              15.51&        0.71&               4.35  &               0.19  &                 0.81                   &        18.95           &         0.03           &  15.33 \\
              17.09&        0.80&               5.93  &               0.28  &                 1.63                   &        35.20           &         0.08           &  17.36 \\
    \midrule
             111.57&        5.25&               0.00  &               0.00  &                 6.40                   &       144.33           &         0.29           & Summen \\
    \midrule
              11.16&        0.52&&&&&& Durchschnitte \\
    \bottomrule
\end{tabular}

Die Steigung der Regressionsgeraden errechnet sich als:
\begin{equation}
    k = \frac{\sum_{i=1}^{10}(x_i-\overline{x})(y_i-\overline{y})}{\sum_{i=1}^{10}(x_i-\overline{x})^2} = \frac{144.33}{6.4} \si{\newton\per\meter} = \SI{22.57}{\newton\per\meter}
\end{equation}

Den Achsenabschnitt $F_0$ erh\"alt man aus:
\begin{equation}
    F_0 = \overline{y} - k \cdot \overline{x} = \SI{11.16}{\newton} - \SI{22.57}{\newton\per\meter} \cdot \SI{0.52}{\meter} = \SI{-0.69}{\newton}
\end{equation}

Die empirische Korrelation betr\"agt:
\begin{equation}
    r_{xy} = \frac{\sum_{1}^{10}(x_i - \overline{x}) \cdot (y_i - \overline{y})}{\sqrt{\sum_{1}^{10}(x_i-\overline{x})^2 \cdot \sum_{1}^{10}(y_i-\overline{y})^2}}
           = \frac{6.40}{\sqrt{144.33 \cdot 0.29}} = 0.99364
\end{equation}

Das Bestimmtheitsmass betr\"agt:
\begin{equation}
    R^{2} = r_{xy}^2 = 0.98732
\end{equation}
\todo{Unsicherheiten: Wie?}

\subsubsection{Taschenrechner}
Ergebnisse ermittelt mittels TI-89:
%(Anleitung: http://home2.fvcc.edu/~erady/Classes/Calc-F07/ti89regression.pdf)
\begin{gather*}
    F = k \cdot z + F_0 \\
    k = \SI{0.044312}{\newton\per\meter} \\
    F_0 = \SI{0.03061}{\newton} \\
    corr = 0.993638 \\
    R^2 = 0.987316
\end{gather*}
